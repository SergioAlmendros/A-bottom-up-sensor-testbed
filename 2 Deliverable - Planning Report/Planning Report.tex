\documentclass[10pt,a4paper]{article}

\title{A bottom up sensor testbed}
\author{Sergio Almendros Diaz}
\date{\today}

\begin{document}
\maketitle

\section{Planning Report}

The following sections explain the tasks that I will do in the course of this project.

\subsection{Small problem solution}

I want to do an easy example to how to connect an arduino with a server running in my computer, what I want to do is mix the two examples of the arduino IDE (Blink and UDPSendReceiveString). The final result should be a program in my computer that communicates with the arduino with an UDP command to light a LED fot 10 seconds.

This is a reduce problem of the real "bottom-up sensor testbed" because, at the end, in every arduino will be a program that will have to send a message to a server with the data of the sensors attached to it.

\subsection{Collect Data from sensors}

First I will connect a temperature sensor to the arduino YUN, then, I will develop a program to collect the information from it, and send it to a small server.
When the temperature sensor works, I will do the same process with a humidity, light, and noise sensor.

\subsection{Install Sentilo}

Sentilo (www.sentilo.io) is an open source sensor and actuator platform that I will install in my laptop to act as the server between the sensor network and the interface for the users to visualize the data. 

\subsection{Communication with Sentilo}

I will adapt the messages that the arduino send to fit with the Sentilo.

\subsection{First big test}

At this moment, the part of the arduino and the server will be done, so I will test the server for the real number of guifi nodes with a program which will emulate x arduinos sending different values of temperature, humidity, light, and noise, to test the capacity of the Sentilo platform.

\subsection{Interface}

I want to do an interface for any user to understand the meaning of the temperature, humidity, light, and noise values. This interface will be develop for an android mobile application.

\subsection{Sentilo module}

I will develop a module for the Sentilo platform, ideally it will be a authentication module, but by the time I could start this task it may be developed, so this task is a little bit unknown.

\subsection{Memory}

This task have to be done in parallel with all the other ones, and its purpose is document all the work that I have done.

\end{document}
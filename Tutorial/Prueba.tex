\documentclass[10pt,a4paper]{article}

\title{A bottom up sensor testbed}
\author{Sergio Almendros Diaz}
\date{\today}

\begin{document}
\maketitle

\section{Pilot Charter}

Fellow: Sergio Almendros Diaz
\\
Mentor: 
\\
Advisor: Jaume Barcelo

\subsection{Pilot purpose or justification}
The purpose of this pilot is to build a sensor platform that can be attached to guifi nodes to gather and share sensory data.

\subsection{Measurable pilot objectives and related success criteria}
\begin{itemize}
\item Gather data about temperature, humidity, light and noise. 
\item Share the data as open data.
\item Deploy at least two nodes and gather data for at least two weeks.
\end{itemize}

\subsection{High-level requirements}
\begin{itemize}
\item Outdoor enclosure.
\item Use open hardware and open software to the possible extent.
\item Use standardized interfaces to integrate with other projects.
\end{itemize}

\subsection{High-level pilot description}
The goal is to use an arduino platform to create a bottom-up broadband wireless sensor networks. 
As guifi.net has already over 20,000 nodes, the idea is to co-locate the sensory platforms together with the guifi.net nodes and use the guifi.net network to transmit the data.
This data should be gathered and shared.
Ideally, the pilot should include a presentation interface for the users to visualize the data.

\subsection{High-level risks}
A possible risk is that the prototypes are not rugged enough for outdoor environments.
It is also a risk that the prototype is not stable and needs to be reset very often.

\subsection{Summary milestone schedule}
\begin{itemize}
\item From 20/09/2013 to 23/09/2013
	\begin{itemize}
	\item Establish the general idea of the TFG and specifics goals.
	\end{itemize}
\item From 23/09/2013 to 11/10/2013
	\begin{itemize}
	\item Specify the tasks to do and make a planning.
	\end{itemize}
\item From 11/10/2013 to 30/10/2013
	\begin{itemize}
	\item Connect first sensors to the Arduino.
	\end{itemize}
\item From 31/10/2013 to 10/01/2014
	\begin{itemize}
	\item Connect to guifi network and upload data to an open data platform.
	\end{itemize}
\item From 10/01/2014 to 01/06/2014
	\begin{itemize}
	\item Integration of sensors and communication aspects.
	\item Install prototypes.
	\item Data sharing and visualization.
    \item Data analysis and evaluation of the testbed.
	\end{itemize}
\item From 02/06/2014 to 30/06/20014
	\begin{itemize}
	\item Preparation of the final memory.
	\end{itemize}
\item From 01/07/2014 to the date of the presentation
	\begin{itemize}
	\item Make the presentation.
	\end{itemize}
\end{itemize}

\subsection{Summary budget}
The cost of this pilot will be approximately 4000 €. This quantity is for the scholarship to the student that will develop this pilot, budget for attending a conference or visiting collaborators, and the purchase of the necessary hardware.


\section{Tutorial}
I want to do an easy example to how to connect an arduino with a server running in my computer, what I want to do is mix the two examples of the arduino IDE (Blink and UDPSendReceiveString). The final result should be a program in my computer that comunicates with the arduino with an UDP command to light a LED fot 10 seconds.

This is a reduce problem of the real "bottom-up sensor testbed" because, at the end I will have a program that will ask all the arduino devices to send the data that they have, stop sending or reset it.



\end{document}